%%%%%%%%%%%%%%%%%%%%%%%%%%%%%%%%%%%%%%%%%%%%%%%%%%%%%%%%%%%%%%%%%%%%%%%%%%%%%%%
% A clean template for an academic CV. This is a short summary version.
%
% Uses tabularx to create two column entries (date and job/edu/citation).
% Defines commands to make adding entries simpler.
%
%%%%%%%%%%%%%%%%%%%%%%%%%%%%%%%%%%%%%%%%%%%%%%%%%%%%%%%%%%%%%%%%%%%%%%%%%%%%%%%

\documentclass[11pt,a4paper]{article}

% Useful aliases
\newcommand{\ICL}{Imperial College London}
\newcommand{\DOM}{Department of Materials}

% Identifying information
\newcommand{\Title}{Curriculum Vit\ae\ Summary}
\newcommand{\FirstName}{Anthony}
\newcommand{\LastName}{Onwuli}
\newcommand{\Initials}{A}
\newcommand{\MyName}{\FirstName\ \LastName}
\newcommand{\Me}{\textbf{\Initials. \LastName}}  % For citations
\newcommand{\Email}{aonwuli@gmail.com}
%\newcommand{\PersonalWebsite}{www.leouieda.com}
%\newcommand{\LabWebsite}{www.compgeolab.org}
\newcommand{\ORCID}{0000-0003-2107-153X}
\newcommand{\GitHubProfile}{AntObi}
\newcommand{\PhoneNumber}{+44 7903098889}
\newcommand{\PersonalWebsite}{antobi.github.io}
% Names for citing coauthors
\newcommand{\Aron}{A. Walsh}
\newcommand{\Keith}{K. T. Butler}
\newcommand{\Alex}{A. M. Ganose}
\newcommand{\Ieuan}{I. Seymour}
\newcommand{\Ainara}{A. Aguadero}
\newcommand{\Ashish}{A.V. Hegde}
\newcommand{\Kevin}{K. Nguyen}
\newcommand{\Hyunsoo}{H. Park}

% Load packages
%%%%%%%%%%%%%%%%%%%%%%%%%%%%%%%%%%%%%%%%%%%%%%%%%%%%%%%%%%%%%%%%%%%%%%%%%%%%%%%

% Full Unicode support for non-ASCII characters
\usepackage[utf8]{inputenc}
\usepackage[english]{babel}
\usepackage[TU]{fontenc}

% Set main fonts
\usepackage[sfdefault]{atkinson}
\usepackage[ttdefault]{sourcecodepro}

% Icon fonts
\usepackage{fontawesome5}
\usepackage{academicons}

% Disable hyphenation
\usepackage[none]{hyphenat}

% Control the font size
\usepackage{anyfontsize}

% For fancy and multipage tables
\usepackage{tabularx}
\usepackage{ltablex}

% For new environments
\usepackage{environ}

% Manage dates and times
\usepackage{datetime}

% Set the page margins
\usepackage{geometry}

% To get the total page numbers (\pageref{LastPage})
\usepackage{lastpage}

% Control spacing in enumerates
\usepackage{enumitem}

% Use custom colors
\usepackage[usenames,dvipsnames]{xcolor}

% Configure section titles
\usepackage{titlesec}

% Fancy header configuration
\usepackage{fancyhdr}

% Control PDF metadata and links
\usepackage[colorlinks=true]{hyperref}


% Template configuration
%%%%%%%%%%%%%%%%%%%%%%%%%%%%%%%%%%%%%%%%%%%%%%%%%%%%%%%%%%%%%%%%%%%%%%%%%%%%%%%

\geometry{%
  margin=15mm,
  headsep=0mm,
  headheight=0mm,
  footskip=5mm,
  includehead=true,
  includefoot=true
}

% Custom colors
\definecolor{mediumgray}{gray}{0.5}
\definecolor{lightgray}{gray}{0.9}
\definecolor{mediumblue}{HTML}{2060c2}
\definecolor{lightblue}{HTML}{a0c3ff}

% No indentation
\setlength\parindent{0cm}

% Increase the line spacing
\renewcommand{\baselinestretch}{1.1}
% and the spacing between rows in tables
\renewcommand{\arraystretch}{1.25}

% Remove space between items in itemize and enumerate
\setlist{nosep}

% Set the spacing and format of sections
\titleformat{\section}
  {\normalfont\Large\mdseries} % format
  {} % label
  {0pt} % separation (left separation for hang)
  {} % text before title
  [\titlerule] % text after title
\titlespacing*{\section}
  {0pt} % left pad
  {0.1cm} % before
  {0cm} % after
\titleformat{\subsection}
  {\normalfont\large\mdseries} % format
  {} % label
  {0pt} % separation (left separation for hang)
  {} % text before title
  [] % text after title
\titlespacing*{\subsection}
  {0pt} % left pad
  {0cm} % before
  {-0.3cm} % after

% Disable number of sections. Use this instead of "section*" so that the sections still
% appear as PDF bookmarks. Otherwise, would have to add the table of contents entries
% manually.
\makeatletter
\renewcommand{\@seccntformat}[1]{}
\makeatother

% Define a new environment to place all CV entries in a 2-column table.
% Left column are the dates, right column the entries.
\newcommand{\TablePad}{\vspace{-0.2cm}}
\NewEnviron{EntriesTableDuration}{
\TablePad
\begin{tabularx}{\textwidth}{@{}p{0.13\textwidth}@{\hspace{0.02\textwidth}}p{0.85\textwidth}@{}}
  \BODY
\end{tabularx}
\TablePad
}
\NewEnviron{EntriesTableYear}{
\TablePad
\begin{tabularx}{\textwidth}{@{}p{0.05\textwidth}@{\hspace{0.01\textwidth}}p{0.94\textwidth}@{}}
  \BODY
\end{tabularx}
\TablePad
}
\NewEnviron{EntriesTableSkill}{
\TablePad
\begin{tabularx}{\textwidth}{@{}p{0.23\textwidth}@{\hspace{0.01\textwidth}}p{0.76\textwidth}@{}}
  \BODY
\end{tabularx}
\TablePad
}

% Macros to set the year and duration on the left column
\newcommand{\Duration}[2]{\fontsize{10pt}{0}\selectfont \texttt{#1-#2}}
\newcommand{\Year}[1]{\fontsize{10pt}{0}\selectfont \texttt{#1}}
\newcommand{\Ongoing}{Present}
\newcommand{\Future}{future}

% Macros to add links and mark publications
\newcommand{\DOI}[1]{DOI:\href{https://doi.org/#1}{#1}}
\newcommand{\Website}[1]{\href{https://#1}{#1}}
\newcommand{\Preprint}[1]{\faFilePdf{} ArXiv: \href{https://doi.org/#1}{#1}}
\newcommand{\ChemPreprint}[1]{\faFilePdf{} ChemrXiv: \href{https://doi.org/#1}{#1}}
\newcommand{\GitHub}[1]{\faGithub{} GitHub: \href{https://github.com/#1}{#1}}
\newcommand{\OA}{\aiOpenAccess{}\ }

% Define command to insert month name and year as date
\newdateformat{monthyear}{\monthname[\THEMONTH], \THEYEAR}

% Configure a fancy footer
\newcommand{\Separator}{\hspace{3pt}|\hspace{3pt}}
\newcommand{\FooterFont}{\footnotesize\color{mediumgray}}
\pagestyle{fancy}
\fancyhf{}
\lfoot{%
  \FooterFont{}
  \MyName{}
  \Separator{}
  \Title{}
}
\rfoot{%
  \FooterFont{}
  Last updated: \monthyear\today{}
  \Separator{}
  \thepage\space of\space \pageref*{LastPage}
}
\renewcommand{\headrulewidth}{0pt}
\renewcommand{\footrulewidth}{1pt}
\preto{\footrule}{\color{lightgray}}

% Metadata for the PDF output and control of hyperlinks
\hypersetup{
  colorlinks,
  allcolors=mediumblue,
  breaklinks=true,
  pdftitle={\MyName},
  pdfauthor={\MyName},
}

%%%%%%%%%%%%%%%%%%%%%%%%%%%%%%%%%%%%%%%%%%%%%%%%%%%%%%%%%%%%%%%%%%%%%%%%%%%%%%%
\begin{document}

\begin{minipage}[t]{0.5\textwidth}
  {\fontsize{20pt}{0}\selectfont\MyName}
\end{minipage}
\begin{minipage}[t]{0.5\textwidth}
  \begin{flushright}
    Email: \href{mailto:\Email}{\Email}
  \end{flushright}
\end{minipage}
\\[-0.1cm]
\textcolor{lightgray}{\rule{\textwidth}{3pt}}
\begin{minipage}[t]{0.5\textwidth}
  Location: {London, UK}
\end{minipage}
\begin{minipage}[t]{0.5\textwidth}
  \begin{flushright}
  Tel: \PhoneNumber{}

  \end{flushright}
\end{minipage}
\vspace{-0.3cm}

%%%%%%%%%%%%%%%%%%%%%%%%%%%%%%%%%%%%%%%%%%%%%%%%%%%%%%%%%%%%%%%%%%%%%%%%%%%%%%%


%%%%%%%%%%%%%%%%%%%%%%%%%%%%%%%%%%%%%%%%%%%%%%%%%%%%%%%%%%%%%%%%%%%%%%%%%%%%%%%

\section{Education and Research Experience}

\begin{EntriesTableDuration}
  \Duration{2024}{\Ongoing} &
  \textbf{EPSRC Doctoral Prize Research Assistant}, \ICL{}, UK.
  \begin{itemize}
    \item Awarded funding of £69,118 for a 12 month fellowship.
  \end{itemize}
  \\
  \Duration{2020}{2024}  &
  \textbf{PhD in Materials Research}, \ICL{}, UK.
  \begin{itemize}
    \item \textit{'Rapid computational screening of materials for energy storage applications'}
    \item Supervised by Aron Walsh and Keith T. Butler (UCL).
    \item Funded by a UKRI EPSRC studentship for 3.5 years.
    \item Adapted natural language processing techniques to create ionic representations for applications in property prediction and automated doping suggestions.
    \item Developed software for analysis and visualisation of high dimensional elemental and ionic representations.
    \item Designed and implemented workflows for high throughput density functional theory calculations and molecular dynamics simulations for finding new solid state electrolytes.
    \item Passed in April 2024
  \end{itemize}
  \\
  \Duration{2016}{2020}  &
  \textbf{MEng in Materials Science and Engineering}, \ICL{}, UK.
  \begin{itemize}
    \item First class honours.
    \item Final year project: \textit{'Rapid structure prediction using structural analogy'}.
  \end{itemize}
\end{EntriesTableDuration}

%%%%%%%%%%%%%%%%%%%%%%%%%%%%%%%%%%%%%%%%%%%%%%%%%%%%%%%%%%%%%%%%%%%%%%%%%%%%%%%
\section{Supervision and Teaching Experience}

\begin{EntriesTableDuration}
  \Duration{2021}{2022} & \textbf{Supervisor}, \ICL{}, UK.
  \begin{itemize}
    \item Supervised two MSc students on a project on applying element representations to crystal structure prediction via structure substitutions.
    \item Assisted and co-supervised a UROP student on a project on applying unsupervised machine learning techniques to phonon data to discover solid electrolytes. 
    \item Web scraped data to provide a repository for the UROP project. \GitHub{WMD-Group/phonondb} 
    \vspace{-\baselineskip}
  \end{itemize}
  \\
  \Duration{2021}{2024} & \textbf{Graduate Teaching Assistant}, \ICL{}, UK.
  \begin{itemize}
    \item Developed Jupyter Notebook teaching materials for new course on machine learning for materials science for master's students.
    \item Delivered workshops for introduction to python courses for first year materials science undergraduates.
    \item Guided second year materials science undergraduates through workshops on machine learning using python.
    \vspace{-\baselineskip}
  \end{itemize}
\end{EntriesTableDuration}

%%%%%%%%%%%%%%%%%%%%%%%%%%%%%%%%%%%%%%%%%%%%%%%%%%%%%%%%%%%%%%%%%%%%%%%%%%%%%%%
\section{Open Source Research Software}

\begin{EntriesTableDuration}
  \Duration{2022}{\Ongoing} &
  \textbf{ElementEmbeddings} | \GitHub{WMD-Group/ElementEmbeddings}
  \newline
  \textit{Python package to analyse high-dimensional representations of the chemical elements using different statistical measures}
  \newline
  Role: Creator and sole developer
  \\
  \Duration{2019}{\Ongoing} &
  \textbf{SMACT} | \GitHub{WMD-Group/SMACT}
  \newline
  \textit{Python package to aid materials design and informatics}
  \newline
  Role: Lead maintainer, developer, GitHub CI/CD setup
\end{EntriesTableDuration}

%%%%%%%%%%%%%%%%%%%%%%%%%%%%%%%%%%%%%%%%%%%%%%%%%%%%%%%%%%%%%%%%%%%%%%%%%%%%%%%
%%%%%%%%%%%%%%%%%%%%%%%%%%%%%%%%%%%%%%%%%%%%%%%%%%%%%%%%%%%%%%%%%%%%%%%%%%%%%%%

%%%%%%%%%%%%%%%%%%%%%%%%%%%%%%%%%%%%%%%%%%%%%%%%%%%%%%%%%%%%%%%%%%%%%%%%%%%%%%%

%%%%%%%%%%%%%%%%%%%%%%%%%%%%%%%%%%%%%%%%%%%%%%%%%%%%%%%%%%%%%%%%%%%%%%%%%%%%%%%


%%%%%%%%%%%%%%%%%%%%%%%%%%%%%%%%%%%%%%%%%%%%%%%%%%%%%%%%%%%%%%%%%%%%%%%%%%%%%%%
\section{Presentations}
\vspace{0.2cm}
\subsection{Talks}

\begin{EntriesTableYear}
  \Year{2023}  &
  \Me, \Alex, \Ainara, \Aron, \Ieuan .
  \textbf{Materials design of quaternary sodium halide electrolytes}
  \emph{RSC SSCG Christmas Meeting, Edinburgh, UK}.
  \\
  \Year{2023}  &
  \Me, \Alex, \Ainara, \Aron, \Ieuan .
  \textbf{Materials design of quaternary sodium halide electrolytes}
  \emph{2023 Fall MRS Meeting, Boston MA, USA}.
  \\
\Year{2023}  &
  \Me, \Ashish, \Kevin, \Keith, \Aron .
  \textbf{A periodic table for the machine learning era}
  \emph{TYC Student Day, University College London, UK}.
  \\
\Year{2023}  &
  \Me, \Ashish, \Kevin, \Keith, \Aron .
  \textbf{Rapid structure prediction}
  \emph{Machine Learning for Materials: Data-driven materials design (2.0), Imperial College London UK}.
  \\
\end{EntriesTableYear}

\subsection{Posters}

\begin{EntriesTableYear}
\Year{2023}  &
  \Me, \Ashish, \Kevin, \Keith, \Aron .
  \textbf{Element similarity in high-dimensional materials representations}
  \emph{CECAM Crystal Structure Prediction workshop, Liverpool, UK}.
  \\
\Year{2023}  &
  \Me, \Alex, \Ainara, \Aron, \Ieuan .
  \textbf{Materials design of quaternary sodium halide electrolytes}
  \emph{RSC MC16, Dublin, Ireland}.
  \\
\Year{2022}  &
  \Me, \Keith, \Aron .
  \textbf{Exploration of the oxide garnet search space}
  \emph{Psi-K 2022, Lausanne, Switzerland}.
  \\
\end{EntriesTableYear}

%%%%%%%%%%%%%%%%%%%%%%%%%%%%%%%%%%%%%%%%%%%%%%%%%%%%%%%%%%%%%%%%%%%%%%%%%%%%%%%
\section{Publications}

\begin{EntriesTableYear}
  \Year{2024} &
  \Hyunsoo, \Me, \Keith, \Aron.
  \textit{Mapping inorganic crystal chemical space}
  \newline
  Faraday Discussions
  \DOI{10.1039/D4FD00063C}
  \\
  \Year{2023}  &
  \Me, \Ashish, \Kevin, \Keith, \Aron.
  \textit{Element similarity in high-dimensional materials representations},
  \newline
  Digital Discovery
  \DOI{10.1039/D3DD00121K}
  \\
\end{EntriesTableYear}
%%%%%%%%%%%%%%%%%%%%%%%%%%%%%%%%%%%%%%%%%%%%%%%%%%%%%%%%%%%%%%%%%%%%%%%%%%%%%%%
\section{Memberships and Committees}

\begin{EntriesTableDuration}
  \Duration{2020}{2024} &
  \textbf{Department of Materials Graduate Society Committee}, \ICL{}, UK.
  \newline
  Role: Cohort representative
  \begin{itemize}
    \item Organised annual postgraduate research days between 2021-2024  including organising speakers, poster sessions and social events.
    \item Handled PhD student feedback on issues surrounding supervising MSc/MEng student which led to an organised workshop for PhD students on supervising and meetings with project coordinators.
    \vspace{-\baselineskip}
  \end{itemize}
  \vspace{-\baselineskip}
\end{EntriesTableDuration}
%%%%%%%%%%%%%%%%%%%%%%%%%%%%%%%%%%%%%%%%%%%%%%%%%%%%%%%%%%%%%%%%%%%%%%%%%%%%%%%
\section{Skills}

\begin{EntriesTableSkill}
  \textbf{Programming} & Preferred languages: \texttt{Python}, other languages: \texttt{Bash}, \texttt{LaTeX}.
  \\
  \textbf{Electronic Structure} & Intermediate user of \texttt{VASP}.
  \\
  \textbf{Materials Informatics} &  Intermediate user of  materials informatics packages including \texttt{matminer} and \texttt{pymatgen}.
  \\
  \textbf{Big Data} & Frequent user of database, data mining, web scraping, machine learning and workflow management tools including \texttt{MongoDB}, \texttt{Pandas}, \texttt{Atomate2}, \texttt{Fireworks}, \texttt{scikit-learn} and \texttt{TensorFlow}.
  \\
  \textbf{Other} & User of \texttt{Git} version control, \texttt{GitHub} CI/CD, Slurm and SGE job scheduling on high performance computing systems, Mac, Linux and Windows operating systems.
\end{EntriesTableSkill}

\end{document}
